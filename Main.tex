\documentclass[xcolor=dvipsnames]{beamer}

\usepackage{xcolor}
\usepackage[ngerman]{babel} % deutsche Silbentrennung
\usepackage[utf8]{inputenc} % wegen deutschen Umlauten
\usepackage{pdfpages}
\usepackage{graphicx}
\usepackage{pst-node}% http://ctan.org/pkg/pst-node
\usepackage{eurosym}


\usetheme{metropolis}           % Use metropolis theme
\title{The smallest grammar problem}
\date{05. Juli 2019}
\author{Edgar Dorausch}
%\institute{Centre for Modern Beamer Themes}
\begin{document}
\maketitle

\newcommand{\Gap}{$ $ \linebreak}
\newcommand{\FrameName}{
	\ifthenelse{\equal{\subsecname}{}}{
		\secname
	}{
		\secname \thinspace -\thinspace\subsecname
	}
}

\newcommand{\Fresh}{\ddagger}
\newcommand{\Hint}[1]{\textcolor{gray}{#1}}


\section{Motivation und Anwendung}

\begin{frame}{\FrameName}
	\begin{itemize}[<+->]
		\item Mustererkennung
		\item Kompression
	\end{itemize}
\end{frame}

\section{Definitionen und Wiederholung}
\begin{frame}{\FrameName}
	\begin{block}{Kontextfreie Grammatik}
		\Gap
		Eine KFG ist ein Quadrupel $(\Sigma,\Gamma,S,\Delta)$ mit
		\begin{itemize}
			%	\item \alert<4>{This is\only<4>{ really} important}
			\item $\Sigma$ - Terminalalphabet
			\item $\Gamma$ - Nichtterminalalphabet
			\item $S$ - Startsymbol
			\item $\Delta$ - Menge von Regeln der Form $T\rightarrow\alpha$\linebreak
			$T \in \Gamma$;
			$\alpha \in (\Sigma \cup \Gamma)^\ast$
		\end{itemize}
	\end{block}
	
\end{frame}

\begin{frame}{\FrameName}
\begin{alert}{Besonderheit!:}
	\Gap
	Die Grammatiken sollen nur ein Wort erzeugen. Deshalb:
	\begin{itemize}
		
		\item Grammatik muss azyklisch sein
		\item Für jedes $T \in \Gamma$ existiert nur eine Regel in $\Delta$
	\end{itemize}
\end{alert}
\end{frame}

\begin{frame}{\FrameName}
\begin{block}{Expansion  eines Strings $\alpha$}
	\Gap
	Erhält man durch erschöpfendes Anwenden der Regeln in einer Grammatik bis nur noch Terminale enthalten sind. \linebreak
	Notation: $\langle \alpha \rangle$
\end{block}
\end{frame}

\begin{frame}{\FrameName}
\begin{block}{Expansionslänge}
	\Gap
	Anzahl der Zeichen in der Expansion eines Strings $\alpha$ \linebreak
	Notation: $[\alpha]  = \lvert \langle \alpha \rangle \lvert$
\end{block}
\end{frame}

\begin{frame}{\FrameName}
\begin{block}{Größe einer Grammatik G}
	\Gap
	Anzahl der Zeichen in den rechten Seiten der Grammatikregeln\linebreak
	Notation: $m = \lvert G \lvert = \sum\limits_{(T \rightarrow \alpha) \in \Delta} \langle \alpha \rangle$ \linebreak $ $\linebreak
	Größe der kleinsten Grammatik für einen String: $m^*$
\end{block}
\end{frame}

\begin{frame}{\FrameName}
\begin{block}{Beispiel}
	$$
	G \colon \begin{Bmatrix} 
		S \rightarrow rhaTber \textvisiblespace TTa \\
		T \rightarrow bar
	\end{Bmatrix}
	$$
	
	$\langle S \rangle = rhabarber \textvisiblespace barbara$ \linebreak
	$[S] = 17$ \linebreak
	$\lvert G \lvert = 11$
\end{block}
\end{frame}

\begin{frame}{\FrameName}
\begin{block}{Approximation Ratio}
	\Gap
	Sei $G_A$ die Grammatik, die von einem Algorithmus $A$ erzeugt wird.
	$$
	a(n) = \alert<2>{\max\limits_{\alpha \in \Sigma^n}}\frac{
		\textrm{$\lvert G_A \lvert$ für $\alpha$}
	}{
		\textrm{$m^*$ für $\alpha$}
	}
	$$
	\begin{center}\alert{
			
			\only<2>{Worstcase!}
		}
	\end{center}
	
\end{block}
\end{frame}

\begin{frame}{\FrameName}
\begin{table}
	\caption{Landau Notation}
	\begin{tabular}{ r p{3.5cm} l}
		
		$f \in o(g)$ & "$f < g$" \\
		%$\only<1>{f \in \mathcal{O}(g)}\only<2>{\alert{f \in \mathcal{O}(g)}}$ & "$f\leq g$" \\
		\textcolor{gray}{(Upper bound)} $f \in \mathcal{O}(g)$ & "$f\leq g$" \\
		$f \in \Theta(g)$ & "$f = g$"\\
		\textcolor{gray}{(Lower bound)} $f \in \Omega(g)$ & "$f \geq g$"\\
		$f \in \omega(g)$ & "$f > g$"\\
	\end{tabular}
\end{table}
\end{frame}

\section{Komplexität}
	
\begin{frame}{\FrameName}
\begin{itemize}[<+->]
	\item Vertex Cover auf SGP reduzieren
	\item Approximationsschranke für polynomielle Algorithmen
	\item Zusammenhang mit Addition Chains \textcolor{gray}{(nicht im Vortrag)}
\end{itemize}
\end{frame}

\begin{frame}{\FrameName}
\begin{block}{Vertex Cover}
	\Gap
	(Minimale) Menge von Knoten, sodass jede Kante mindestens einen dieser Knoten enthält.\linebreak
	$ $\linebreak
	
	\only<1>{\includegraphics[width=0.25\textwidth]{Images/VertexCover/blank}}
	\only<2>{\includegraphics[width=0.25\textwidth]{Images/VertexCover/wrong}}
	\only<3>{\includegraphics[width=0.25\textwidth]{Images/VertexCover/wrongMarked}}
	\only<4>{\includegraphics[width=0.25\textwidth]{Images/VertexCover/right}}
	\visible<3>{\alert{(Kein Vertex Cover!)}}
\end{block}
\end{frame}

\newcommand{\ExampleGraphV}{V = \{a,b,c,d\}}
\newcommand{\ExampleGraphE}{E = 
	\begin{Bmatrix}
		\{a,b\},
		\{a,c\},
		\{b,c\},
		\{b,d\}
	\end{Bmatrix}
}

\begin{frame}{\FrameName}
\begin{block}{NP-härte}
	\begin{itemize}[<+->]
			\item Graphen mit maximalen Knoten-Grad 3
			\item Graph als Wort \Hint{(Unbeschränktes Alphabet)}
			\item Kleinsete Grammatik $\rightarrow$ Vertex Cover
			\item Upper bound für effiziente Approximation \linebreak \Hint{(außer $P=NP$)}
		\end{itemize}
\end{block}
\end{frame}

\begin{frame}{\FrameName}
\begin{block}{Beispiel Graph}
	\Gap
	$$\ExampleGraphV$$ 
	$$\ExampleGraphE$$
	\begin{center}
		\includegraphics[width=0.35\textwidth]{Images/VertexCover/blank}
	\end{center}
\end{block}
\end{frame}

\newcommand{\ProdRuleOne}[1]{(\# #1 \Fresh #1\#  \Fresh )^2}
\newcommand{\ProdRuleTwo}[1]{\# #1 \#  \Fresh}
\newcommand{\ProdRuleThree}[2]{\# #1 \# #2\#  \Fresh}
\newcommand{\PhantomAlpha}{\phantom{\alpha_{Beispiel} = (}}
\newcommand{\ReductionExample}{
	$
		\alpha_{Beispiel} =
		\textcolor{OrangeRed}{
			\foreach \n in {a,b,c,d}{\ProdRuleOne{\n}}
		} \linebreak
		\PhantomAlpha
		\textcolor{PineGreen}{
			\foreach \n in {a,b,c,d}{\ProdRuleTwo{\n}}
		} \linebreak
		\PhantomAlpha
		\textcolor{RoyalBlue}{
			\ProdRuleThree{a}{b}
			\ProdRuleThree{a}{c}
			\ProdRuleThree{b}{c}
			\ProdRuleThree{b}{d}
		} \linebreak
	$
}

\begin{frame}{\FrameName}
\begin{block}{Graphen zu String überführen}
	\PDFC{Double-Dagger = fresh symbol}
	\Gap
	$
	\alpha =
	\textcolor{OrangeRed}{
		\prod\limits_{v_i \in V}\ProdRuleOne{v_i}}
	\textcolor{PineGreen}{
		\prod\limits_{v_i \in V}(\ProdRuleTwo{v_i})}
	\textcolor{RoyalBlue}{
			\prod\limits_{\{v_i,v_j\} \in E}(\ProdRuleThree{v_i}{v_j})}
	$

	
	\visible<2>{
		\Gap
			$\ExampleGraphV; \ExampleGraphE$ \newline
		\Gap
		\ReductionExample
	}
\end{block}
\end{frame}

\begin{frame}{\FrameName}
	\ReductionExample
\begin{block}{Eigenschaften der kleinsten Grammatik}
	\begin{itemize}[<+->]
		\item Jedes Nichtterminal expandiert zu $\#v_i$, $v_i\#$ oder $\#v_i\#$
		\item Enthält Regeln der Form $T_j \rightarrow \#v_i$ und $T_j \rightarrow v_i\#$
		\item \alert<4>{$C = \{v_i \in V | \exists T_j \rightarrow \#v_i\#\}$ ist (minimale) Vertex Cover}
		\item[] \hfill $\qed$
	\end{itemize}
\end{block}
\end{frame}

\begin{frame}{\FrameName}
\begin{block}{Approximation Ratio}
	\begin{itemize}[<+->]
		\item \alert<5>{$m^* = 15|V| + 3|E| + |C|$}
		\item \alert<6>{Vertex Cover kleiner als $\frac{145}{144} \cdot |C|$ finden ist ($NP$) hart} \Hint{($\frac{145}{144}\approx 1,006944...$)}
		\item \alert<7>{$\frac{3}{2}|V| \ge |E|$ \& $\frac{1}{3}|V| \le |C|$}
		% \item $\rho = \frac{15|V| + 3|E| + \frac{145}{144}|C|}{15|V| + 3|E| + |C|}$
		% \item $\rho \geq \frac{15|V| + 3 \cdot \frac{3}{2}|V| + \frac{145}{144}(\frac{1}{3}|V|)}{15|V| + 3 \cdot \frac{3}{2}|V| + \frac{1}{3}|V|} = \frac{8569}{8568} \approx 1,0001167...$
		\item[] \only<-4>{$$
			a(n) = \max\limits_{\alpha \in \Sigma^n} \frac{m}{m^*}
		$$}
		\only<5>{$$
			a(n) = \max\limits_{\alpha \in \Sigma^n} \frac{m}{\alert{15|V| + 3|E| + |C|}}
		$$}
		\only<6>{$$
			a(n) \alert{\ge} \max\limits_{\alpha \in \Sigma^n} \frac{\alert{15|V| + 3|E| + \frac{145}{144}|C|}}{15|V| + 3|E| + |C|}
		$$}
		\only<7>{$$
			a(n) \ge \max\limits_{\alpha \in \Sigma^n} \frac{15|V| + 3 \alert{\cdot \frac{3}{2}|V|} + \frac{145}{144}(\alert{\frac{1}{3}|V|})}{15|V| + 3 \alert{\cdot \frac{3}{2}|V|} + \alert{\frac{1}{3}|V|}}
		$$}
		\only<8>{$$
			a(n) \ge \max\limits_{\alpha \in \Sigma^n} \alert{\frac{8569}{8568}}
		$$}
		\only<9>{$$
			a(n) \ge \frac{8569}{8568} \approx 1,0001167...
		$$}

		\PDFC{warum dritte regel anwendbar? => ausdruk minimal wen...; erklären dass polynom. algorithmus betrachtet wird}

		
	\end{itemize}
\end{block}
\end{frame}

\section{Algorithmen}

\subsection{Vorüberlegungen}

\newcommand{\LowerBound}{\textcolor{TealBlue}{f_l(n)}}
\newcommand{\UpperBound}{\textcolor{Salmon}{f_u(n)}}

\begin{frame}{\FrameName}
\begin{block}{Lower bound bestimmen}
	\begin{itemize}[<+->]
		\item Definiere $\alpha$ \textcolor{gray}{(n = $|\alpha |$)}
		\item Bestimme lower bound von $m$ \linebreak \textcolor{gray}{$m \in \Omega(\LowerBound)$}
		\item Bestimme upper bound von $m^*$ \linebreak \textcolor{gray}{$m^* \in \mathcal{O}(\UpperBound)$}
	\end{itemize}
	\only<4>{
		$\Rightarrow$
		\fbox{
		$a(n) \in \Omega(\frac{
			\LowerBound
		}{
			\UpperBound
		})$
		}}
\end{block}
\end{frame}

\subsection{LZ78}

\begin{frame}{\FrameName}
	\begin{block}{LZ78}
		\begin{itemize}[<+->]
			\item Gänginger Kompressionsalgorithmus
			\item Abraham Lempel und Jacob Ziv (1978)
			\item Verwendung bei GIF und TIFF \PDFC{Erweiterung wird verwendet}
		\end{itemize}
	\end{block}
	\end{frame}

\begin{frame}{\FrameName}
\begin{block}{LZ78 - Datenstrukturen}
	\begin{itemize}[<+->]
		\item Strings als Sequenzen von Paaren $(i,c)$ dargestellt \linebreak \Hint{$i$...Index eines Vorgänger-Paares oder $0$$; c \in \Sigma$}
		\item Jedes Paar repräsentiert Substring
		\item Wenn $i$ gleich $0$ dann ist dieser Substring gleich $c$
		\item Andernfalls ist der Substring des $i$-ten Paares gefolgt von $c$
	\end{itemize}
\end{block}
\end{frame}

\begin{frame}{\FrameName}
\begin{block}{Beispiel}
	\Gap
	\Gap
	\begin{center}
		\only<1>{
			\includegraphics[width=0.8\textwidth]{Images/LZ78/blank}}
		\only<2>{
			\includegraphics[width=0.8\textwidth]{Images/LZ78/withRefs}}
		\only<3>{
			\includegraphics[width=0.8\textwidth]{Images/LZ78/full}}
	\end{center}
\end{block}
\end{frame}

\begin{frame}{\FrameName}
	\begin{block}{LZ78 Grammatiken}
		\begin{itemize}[<+->]
			\item Paar $\hat{=}$ Nichterminal
			\item $\begin{cases}
				X_j \rightarrow c, & i = 0\\
				X_j \rightarrow X_i c, & \text{sonst}
			\end{cases}$
			\item $S \rightarrow X_1 ... X_k$
			
		\end{itemize}
		\visible<4>{
			\begin{minipage}[t]{0.5\textwidth}
				$S \rightarrow \foreach \n in {1,2,3,4,5,6}{X_{\n}}$ 
				
	
				$X_1 \rightarrow a; X_2 \rightarrow X_1b;  X_3 \rightarrow b$
				$X_4 \rightarrow X_2a; X_5 \rightarrow X_3a; X_4 \rightarrow X_6\text{\euro}$
			\end{minipage}
			\begin{minipage}[c]{0.48\textwidth}
				\includegraphics[width=\textwidth]{Images/LZ78/full}
			\end{minipage}
		}
	
	\end{block}
	\end{frame}

\begin{frame}{\FrameName}
\begin{block}{LZ78 - Algorithmus}
	\begin{itemize}[<+->]
		\item String wird Schrittweise in einem Durchlauf von links nach rechts in eine Sequenz von Paaren übersetzt
		\item Finde in jedem Schritt das kürzeste Präfix $\gamma$ des verbleibenden Strings das nicht Expansion eines bereits erzeugten Paars ist
		\item Am Ende des Strings muss eventuell ein weiteres Zeichen hinzugefügt werden
		\item Ein neues Paar wird an die Liste angehangen:
		\begin{enumerate}
			\item<4-> Wenn da $\gamma = 1$ ist füge $(0,\gamma)$ hinzu
			\item<5-> Andernfalls ist $\gamma = \alpha c$. \linebreak $\alpha$ ... Expansion eines Paars mit dem Index $i_\alpha$ \linebreak $\Rightarrow$ Paar: $(i,c)$
		\end{enumerate}
	\end{itemize}
\end{block}
\end{frame}

% Marking string sequence
\newcommand{\M}[1]{\textcolor{OrangeRed}{#1}}

\begin{frame}{\FrameName}
\begin{block}{Beispiel}
	\begin{description}[<+->]
		\item \M{a}abbababaab\euro
		\item $\underbrace{(0,a)}_{a}$ \M{ab}bababaab\euro
		\item $\underbrace{(0,a)}_{a}$ $\underbrace{(1,b)}_{ab}$ \M{b}ababaab\euro
		\item $\underbrace{(0,a)}_{a}$ $\underbrace{(1,b)}_{ab}$ $\underbrace{(0,b)}_{b}$ \M{aba}baab\euro
		\item $\underbrace{(0,a)}_{a}$ $\underbrace{(1,b)}_{ab}$ $\underbrace{(0,b)}_{b}$ $\underbrace{(2,a)}_{aba}$ \M{ba}ab\euro
		\item $\underbrace{(0,a)}_{a}$ $\underbrace{(1,b)}_{ab}$ $\underbrace{(0,b)}_{b}$ $\underbrace{(2,a)}_{aba}$ $\underbrace{(3,a)}_{ba}$ \M{ab\euro}
		\item $\underbrace{(0,a)}_{a}$ $\underbrace{(1,b)}_{ab}$ $\underbrace{(0,b)}_{b}$ $\underbrace{(2,a)}_{aba}$ $\underbrace{(3,a)}_{ba}$ $\underbrace{(2,\text{\euro})}_{ab\text{\euro}}$
	\end{description}
\end{block}
\end{frame}

\begin{frame}{\FrameName}
\begin{block}{LowerBound}
	\begin{center}
		\fbox{$ \alpha_k = a^{k(k+1)/2}(ba^k)^{(k+1)^2} $}
	\end{center}
	\begin{itemize}[<+->]
		\item $|\alpha_k | = k \frac{k+1}{2} + (1+k)(k+1)^2$ \linebreak
			$\phantom{|\alpha_k |} = k^3 + \frac{7}{2}k^2 + \frac{7}{2}k + 1$
		\item $ n = |\alpha_k | \in \Theta(k^3)$
	\end{itemize}
\end{block}
\end{frame}

\begin{frame}{\FrameName}
\begin{block}{UpperBound $m^*$}
	\begin{center}
		\fbox{$ \alpha_k = a^{k(k+1)/2}(ba^k)^{(k+1)^2} $}
	\end{center}
	\begin{itemize}[<+->]
		\item $m^* \in \mathcal{O}(1+ log(\frac{k^2+k}{2}) + log(k+1)^2 + 1+ log(k))$
		\item $m^* \in \mathcal{O}(log \thinspace k)$
		\item $m^* \in \mathcal{O}(log \thinspace n^\frac{1}{3}) = \mathcal{O}(log \thinspace n)$
	\end{itemize}
\end{block}
\end{frame}

\begin{frame}{\FrameName}
\begin{block}{LowerBound m}
	\begin{center}
		\fbox{$ \alpha_k = a^{k(k+1)/2}(ba^k)^{(k+1)^2} $}
	\end{center}
	\begin{itemize}[<+->]
		\item String wird in zwei Phasen in eine Paar-Sequenz übersetzt
		\item Erste Phase: alle Strings $a...a^k$ zu Paaren übersetzt
		\item Zweite Phase: $a^iba^j$ für alle $i,j \in [0,k]$ wird ein Paar erstellt
		\item $m \in \Omega(\sum_{z=1}^k z + (k+1)^2) = \Omega(k^2)$
		\item $m \in \Omega(n^{2/3})$
	\end{itemize}
\end{block}
\end{frame}

\begin{frame}{\FrameName}
\begin{block}{LowerBound}
	\begin{center}
		\fbox{$ \alpha_k = a^{k(k+1)/2}(ba^k)^{(k+1)^2} $}
	\end{center}
	\begin{itemize}[<+->]
		\item $m^* \in \mathcal{O}(log \thinspace n)$
		\item $m \in \Omega(n^{2/3})$
		\item $a(n) \in \Omega(\frac{n^{2/3}}{log \thinspace n})$
	\end{itemize}
\end{block}
\end{frame}

\begin{frame}{\FrameName}
	\begin{block}{Upper bound}
		\visible<1->{
			\includegraphics[width=0.98\textwidth]{Images/LZ78/ub_blank}}
			\Gap
		\visible<2->{
			\includegraphics[width=0.98\textwidth]{Images/LZ78/ub_sort}}
			\Gap
		\visible<3->{
			\includegraphics[width=0.98\textwidth]{Images/LZ78/ub_underestimate}}
	\end{block}
	\end{frame}

	\begin{frame}{\FrameName}
		\begin{block}{Upper bound}
			\visible<1->{
				\includegraphics[width=0.98\textwidth]{Images/LZ78/ub_blank}}
				\Gap
			\visible<2->{
				\includegraphics[width=0.98\textwidth]{Images/LZ78/ub_sort}}
				\Gap
			\visible<3->{
				\includegraphics[width=0.98\textwidth]{Images/LZ78/ub_underestimate}}
		\end{block}
		\end{frame}

\subsection{global algorithms}
\begin{frame}{\FrameName}
	\begin{block}{TODO}
	\Gap
	$a^2 + b^2 = c^2$
\end{block}
\end{frame}

\subsection{LZ77 variant}
\begin{frame}{\FrameName}
	\begin{block}{TODO}
	\Gap
	Basiert anscheinend auf $BB[\alpha]-Trees$
\end{block}
\end{frame}

\end{document}