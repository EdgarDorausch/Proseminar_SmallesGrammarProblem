\section{Definitionen und Wiederholung}
\begin{frame}{\FrameName}
	\begin{block}{Kontextfreie Grammatik}
		\Gap
		Quadrupel $(\Sigma,\Gamma,S,\Delta)$ mit
		\begin{itemize}
			%	\item \alert<4>{This is\only<4>{ really} important}
			\item $\Sigma$ - Terminalalphabet
			\item $\Gamma$ - Nichtterminalalphabet
			\item $S$ - Startsymbol
			\item $\Delta$ - Menge von Regeln der Form $T\rightarrow\alpha$\linebreak
			$T \in \Gamma$;
			$\alpha \in (\Sigma \cup \Gamma)^\ast$
		\end{itemize}
	\end{block}
	
\end{frame}

\begin{frame}{\FrameName}
\begin{alert}{Besonderheit:}
	\Gap
	Grammatiken sollen nur ein Wort erzeugen \Hint{(straight-line grammar)}
	\begin{itemize}
		
		\item Grammatik azyklisch
		\item Für jedes $T \in \Gamma$ existiert nur eine Regel
	\end{itemize}
\end{alert}
\end{frame}

\begin{frame}{\FrameName}
\begin{block}{Expansion  eines Strings $\alpha$}
	\Gap
	Erschöpfendes Anwenden der Regeln \linebreak
	Notation: \alert{$\langle \alpha \rangle$}

	\visible<2->{
		\Gap
		\textbf{Größe einer Grammatik G}
		\Gap
		Anzahl der Zeichen in den rechten Seiten der Grammatikregeln\linebreak
		Notation: $\alert{m} = \lvert G \lvert = \sum\limits_{(T \rightarrow \alpha) \in \Delta} | \alpha |$}
		\visible<3>{
		\linebreak
		\Gap
		\textbf{Größe der kleinsten Grammatik G}
		\Gap
		Notation: \alert{$m^*$}}
\end{block}
\end{frame}

\begin{frame}{\FrameName}
\begin{block}{Beispiel}
	$$
	G \colon \begin{Bmatrix} 
		S \rightarrow rhaTber \textvisiblespace TTa \\
		T \rightarrow bar
	\end{Bmatrix}
	$$
	
	$\langle S \rangle = rhabarber \textvisiblespace barbara$ 
	\visible<2->{ \linebreak $| \langle S \rangle| = 17$}
	\visible<3->{ \linebreak $\lvert G \lvert = 11$}
\end{block}
\end{frame}

\begin{frame}{\FrameName}
\begin{block}{Approximation Ratio}
	\Gap
	$$
	a(n) = \alert<2>{\max\limits_{\alpha \in \Sigma^n}}\frac{
		\textrm{m für $\alpha$}
	}{
		\textrm{$m^*$ für $\alpha$}
	}
	$$
	\begin{center}\alert{
			
			\visible<2>{Worstcase!}
		}
	\end{center}
	
\end{block}
\end{frame}

\begin{frame}{\FrameName}
\begin{table}
	\caption{Landau Notation}
	\begin{tabular}{ r p{3.5cm} l}
		
		$f \in o(g)$ & "$f < g$" \\
		%$\only<1>{f \in \mathcal{O}(g)}\only<2>{\alert{f \in \mathcal{O}(g)}}$ & "$f\leq g$" \\
		\textcolor{gray}{(Upper bound)} $f \in \mathcal{O}(g)$ & "$f\leq g$" \\
		$f \in \Theta(g)$ & "$f = g$"\\
		\textcolor{gray}{(Lower bound)} $f \in \Omega(g)$ & "$f \geq g$"\\
		$f \in \omega(g)$ & "$f > g$"\\
	\end{tabular}
\end{table}
\end{frame}