
\subsection{Global Algorithms}
\begin{frame}{\FrameName}
	\begin{block}{Global Algorithms}
    \begin{itemize}[<+->]
      \item Klasse von Algorithmen
      \item Haben alle ein upper bound von $\mathcal{O}((\frac{n}{log(n)})^{\frac{2}{3}})$
      \item Lower bounds sind sehr schlecht
    \end{itemize}
\end{block}
\end{frame}

\begin{frame}{\FrameName}
	\begin{block}{Global Algorithms - Verfahren}
    \begin{itemize}[<+->]
      \item Grammatik schrittweise verbessert
      \item Initialisiere Grammatik mit $S \rightarrow \alpha$
      \item Wähle einen String $\gamma$
      \item Füge $T \rightarrow \gamma$ \Hint{(T... neues Nichtterminal)}
      \item Traversiere alle anderen Regeln von links nach rechts und ersetze vorkommen von $\gamma$ durch $T$
    \end{itemize}
\end{block}
\end{frame}

\begin{frame}{\FrameName}
	\begin{block}{Auswahl von $\gamma$}
    \begin{itemize}[<+->]
      \item $|\gamma| \ge 2$
      \item $\gamma$ kommst mind. zwei mal in Grammatik vor \linebreak \Hint{(ohne Überschneidung)}
      \item Alle Strings länger als $\gamma$ kommen seltener vor
    \end{itemize}
\end{block}
\end{frame}

\begin{frame}{\FrameName}
	\begin{block}{Upper Bound}
    \begin{itemize}[<+->]
      \item Ähnlich upper bound von LZ78
      \item Auflistung von Substrings der Länge 2
      \item Einordnung in Gruppen
      \item Abschätzen der Gesamt-Expansionslänge der Gruppen
    \end{itemize}
\end{block}
\end{frame}

\begin{frame}{\FrameName}
	\begin{block}{Upper Bound (1/4)}
    \begin{itemize}[<+->]
      \item Wähle $\frac{2}{9}m$ Substrings der Länge 2 \linebreak \Hint{(ohne Überschneidung)}
      \item ist immer möglich
    \end{itemize}
\end{block}
\end{frame}

\begin{frame}{\FrameName}
	\begin{block}{Upper Bound (2/4)}
    \begin{itemize}[<+->]
      \item Sortiere Substrings aufsteigend nach deren Expansionslänge
      \item Füge die ersten $2m^*$ Substrings der ersten Gruppe hinzu
      \item Füge die nächsten $3m^*$ Substrings der zweiten Gruppe hinzu
      \item usw. ...(bis zur Gruppe mit $gm^*$ Elementen)
      \item $2m^* + 3m^* + ... + gm^* (g+1)m^* > \frac{2}{9}m$
      \item $m^*\sum_{k=2}^{g+1}k = m^* (\frac{g^2}{2} + \frac{3g}{2}) > \frac{2}{9}m$
      \item $m\in \mathcal{O}(g^2 m^*)$
    \end{itemize}
\end{block}
\end{frame}

\begin{frame}{\FrameName}
	\begin{block}{Upper Bound (3/4)}
    \begin{itemize}[<+->]
      \item Sei $\sigma = $ "Gesamt-Expansionslänge"
      \item Für jedes $\alpha$ in i-ten Gruppen gilt: $|\langle \alpha \rangle | \ge i+1$ \Hint{(mk-Lemma)}
      \item $2^2m^* + 3^2m^* + ... + g^2m^* \le \sigma$
      \item $\sigma \le 2n$
      \item $2^2m^* + 3^2m^* + ... + g^2m^* \le 2n$
      \item $m^* \sum_{k=2}^{g}k^2 = m^* (\frac{g^3}{3} + \frac{g^2}{2} + \frac{g}{6} - 1) \le 2n$
      \item $g^3 \in \mathcal{O}(\frac{n}{m^*}) \Rightarrow g \in \mathcal{O}((\frac{n}{m^*})^{\frac{1}{3}})$
    \end{itemize}
\end{block}
\end{frame}

\begin{frame}{\FrameName}
	\begin{block}{Upper Bound (4/4)}
    \begin{itemize}[<+->]
      \item $m\in \mathcal{O}(g^2 m^*)$ und $g \in \mathcal{O}((\frac{n}{m^*})^{\frac{1}{3}})$
      \item $m \in \mathcal{O}((\frac{n}{m^*})^{\frac{2}{3}} m^*) = \mathcal{O}((\frac{n}{log (n)})^{\frac{2}{3}} m^*)$
      \item $a(n) = \max\limits_{\alpha \in \Sigma^n} \frac{m}{m^*} \in \mathcal{O}((\frac{n}{log(n)})^{\frac{2}{3}}) \qed$
    \end{itemize}
\end{block}
\end{frame}

\begin{frame}{\FrameName}
	\begin{block}{Greedy}
    \Gap
    In jeder Iteration wird das $\gamma $ gewählt welches die Größe der Grammatik am meisten senkt.
\end{block}
\end{frame}

\begin{frame}{\FrameName}
	\begin{block}{Lower bound (Beweisskizze)}
    \begin{itemize}[<+->]
      \item $\alpha = x^n$ mit $n = 5^{2^k}$
      \item $m = 5 \cdot 2^k$
      \item $m^* = 3log_3(n) + o(log(n)) $
      \item \only<4>{$a(n) = \max\limits_{\alpha \in \Sigma^n} \frac{m}{m^*}$}
      \only<5>{$a(n) \ge \frac{5 \cdot 2^k}{3log_3(5^{2^k})}$}
      \only<6>{$a(n) \ge \frac{5 log(3)}{3log(5)}$}
      \only<7>{$a(n) \ge \frac{5 log(3)}{3log(5)} \approx 1,137...$ \Hint{(Konstant!)}}
    \end{itemize}
\end{block}
\end{frame}
